\documentclass[a4paper]{article}


\title{Robocup Junior Registration}
\author{Monika Buchalová, Filip Kotoč, Oliver Sidor, Filip Sršeň}
\usepackage{hyperref}

\begin{document}
	\maketitle
	
	\tableofcontents
	\newpage
	\section{Introduction}
	\label{sec:Intro}
	\subsection{Purpose of requirements document}
	\label{sub:purpose}
	Created as a requirements catalog for Information Systems Development project Robocup Junior Slovakia registration in academic year 2023/24. Its purpose is to provide a set of requirements of the information system. 
	\subsection{Scope of the product}
	\label{sub:scope}
 	The project is a web application that provides registration service for Robocup competition, results and documents sharing. 
	\subsection{Definitions, acronyms and abbreviations}
	\label{sub:definitios}
	\subsection{References}
	\label{sub:references}
	\begin{itemize}
		\item 	\href{https://github.com/TIS2023-FMFI/robocup}{GitHub repository of the project }
		\item \href{https://www.skse.sk/}{Current robocup website}
	\end{itemize}
	\subsection{Overview of the remainder of the document}
	\label{sub:overview}
	In the following chapters, the reader will learn about the various functionalities, detailed requirements and will get an idea of how users can work with the web application.
	
	
	\newpage
	
	\section{General description}
	\label{sec:general-desc}
	\subsection{Product perspective}
	\label{sub:perspective}
	Web application will be used for registration of individual competitors, assembling teams and registering them to different categories. From point of view of an organizer there will be a check-in functionality, categories management and event administration.
	\subsection{Product functions}
	\label{sub:functions}
	\subsubsection{Team leader registration}
	Creation of team leader account where user can managed individuals and teams. 
	\subsubsection{Individual registration}
	Team leader can add individuals under their registration. Then are prompted to fill out a form with the personal details.
	\subsubsection{Team assembly}
	Team leader can create and assemble teams out of registered individuals.
	\subsubsection{Category assignment}
	Team leader can assign team to different categories.
	\subsubsection{Organizer registration}
	Organizer fills out a default registration form with password and gets send email to confirm their email address.
	\subsubsection{Check-in}
	On site check-in for teams and individuals done by organizer.
	\subsubsection{Results}
	Creating results pages for each category.
	\subsubsection{Diploma generation}
	Generating diplomas from template with current details.
	\subsubsection{Posting documents}
	Posting rules, propositions etc.
	\subsection{User characteristics}
	\label{sub:users}
	\subsubsection{Team leader}
	Team leader is user who registers a team or multiple teams. Core functionalities for team leaders are:
	\begin{itemize}
		\item Creating a registration.
		\item Registering individual competitors.
		\item Assembling teams from registered individuals.
		\item Registering teams to categories. 
	\end{itemize} 

	\subsubsection{Organizer}
	Organizer is user that can do all the actions necessary for running the event. These are:
	\begin{itemize}
		\item Teams check-in on site
		\item Accessing detailed team information
		\item Updating final scores
		\item Creating diplomas
	\end{itemize}

	\subsubsection{Administrator}
	Administrator is an all powerful user. They can create organizer accounts, setup events and edit event details.
	\begin{itemize}
		\item Create events
		\item Edit event details
		\item Create organizer accounts
	\end{itemize}

	\subsubsection{Viewer}
	This is any user that views the website publicly. They can see list of participants, categories and results. 

	\subsection{General constraints and dependencies}
	\label{sub:constraints}
	The project is an web applications. Users will need only a web browser and internet connection in order to access the application.
	
	\newpage
	
	\section{Specific requirements}
	\label{sec:specific}
	
	User vyplni formular obsahujuci formular, kde vyplni meno, email. Potom dostane na mailom unikatny link/kod registracie, pod ktorym bude vediet upravovat svoju registraciu.
\end{document}

